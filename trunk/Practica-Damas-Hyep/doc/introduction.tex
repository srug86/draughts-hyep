
El objetivo de la práctica es conocer diferentes metodologías, tecnologías y
patrones para el diseño y construcción de softwares. Para ello se va
a realizar una aplicación que implementa el ya conocido juego de las damas. 

Se pretende que la aplicación permita a los usuarios jugar tanto en modo local 
como en modo red, siendo esta última una conexión uno a uno entre dos jugadores
o usuarios. Por lo tanto, el desarrollo de la aplicación se descompone en las 
siguientes etapas:

\begin{itemize}
\item \textbf{Definición de requisitos}: se realiza una análisis de las 
funcionalidades de la aplicación que tiene como resultado la lista de 
requisitos.
\item \textbf{Diseño estructural y funcional}: se realizan diversos diagramas a
partir de los requisitos finales para así entender mejor el funcionamiento de la
aplicación.
\item \textbf{Implementación}: se desarrolla un primer prototipo y a partir de 
éste se procede a implementar la aplicación completa.
\end{itemize}

Con la realización de esta práctica se pretende conocer y asimilar los 
objetivos básicos de \emph{Herramientas y Entornos de Programación}:

\begin{itemize}
\item Utilizar adecuadamente herramientas, lenguajes y entornos de programación
como \emph{Visual Studio}, \emph{Expression Blend}, \emph{.NET} y \emph{C\#}.
\item Adquirir una cierta cultura general respecto de métodos y tecnologías
para el desarrollo del software para las herramientas anteriormente 
mencionadas.
\end{itemize}
