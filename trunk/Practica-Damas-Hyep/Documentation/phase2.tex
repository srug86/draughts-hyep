\section{Herramientas utilizadas}
Para realizar el diseño estructural y funcional de la aplicación se analizaron
herramientas como: \emph{Kivio}, \emph{UML Studio} o \emph{Visual Paradigm for
UML}. Al final, se decidió que \emph{Visual Paradigm for UML} sería la
herramienta más idónea para realizar esta fase.

\subsection{\emph{Visual Paradigm for UML}}
Es una herramienta \emph{CASE} multiplataforma que soporta \emph{UML 2},
\emph{SysML} y la notación y modelado de procesos de negocio (\emph{BPMN}).

\subsubsection{Características}
\emph{Visual Paradigm} posee las siguientes características:
\begin{itemize}
\item Diagramas de diseño automático sofisticado. 
\item Entorno de modelado visual. 
\item 13 tipos de diagramas UML (incluidos de clases y secuencia). 
\item Posibilidad de generar documentación y reportes \emph{UML}.
\item Generación automática de código en multitud de lenguajes de programación.
\item Ingeniería inversa.
\item Licencia propietaria (con una edición libre: la \emph{Free Community
Edition}).
\end{itemize}

\subsubsection{Motivos de la elección}
Se eligió esta herramienta por los siguientes motivos:
\begin{itemize}
\item Años de experiencia, utilizada en otras asignaturas. 
\item Gran poder descriptivo, mejor diseño y mejor implementación que otras 
herramientas analizadas. 
\item Permite la generación automática de código en \emph{C\#} a partir de los
diagramas diseñados.
\item Permite realizar ingeniería inversa. 
\item La ESI nos facilita una licencia para poder utilizar la edición
\emph{VP-UML Standard Edition}.
\end{itemize}

\section{Diagramas elaborados}
Con esta herramienta se elaboró el diagrama de clases de la aplicación y los
principales diagramas de secuencia.

\subsection{Diagramas de clases}
\imagen{Classes.pdf}{12}{Diagrama de clases}{img:Classes}

\imagen{Presentation.pdf}{12}{Clases del paquete \emph{Presentation}}
{img:Presentation}

\imagen{Domain.pdf}{12}{Clases del paquete \emph{Domain}}{img:Domain}

\imagen{Communications.pdf}{12}{Clases del paquete \emph{Communications}}
{img:Communications}

\subsection{Diagramas de secuencia}

%\section{Cambios realizados}

