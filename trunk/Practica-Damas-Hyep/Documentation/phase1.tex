\section{Herramientas utilizadas}
Para realizar la definición de requisitos se barajaron herramientas como:
\emph{Rational RequisitePro}, \emph{Visual Paradigm for ULM} y \emph{Gantter}.
Finalmente, se decidió que esta última sería la más adecuada para realizar
nuestro cometido.

\subsection{\emph{Gantter}}
Es una herramienta libre de gestión de proyectos basada en web que está
disponible en la página \url{http://gantter.com}.

\subsubsection{Características}
\emph{Gantter} posee las siguientes características:
\begin{itemize}
\item Herramienta de planificación de proyectos.
\item Permite establecer los plazos, los recursos, los costes y las tareas de
nuestro proyecto, así como generar con éstos \emph{diagramas de gantt}, 
calendarios, establecer la ruta crítica, etc.
\item Basada en web. Por lo que no hace falta instalar ningún software en
nuestra máquina. Para poder ejecutar la aplicación, la propia página web
ofrece los plugins necesarios para los principales navegadores: \emph{Chrome},
\emph{Firefox}, \emph{Explorer}, \emph{Opera} y \emph{Safari}.
\item Puede integrarse a \emph{Google Docs}. Esto permite que la información
pueda compartirse y actualizarse de forma colaborativa y distribuida.
\item Permite importar proyectos creados en \emph{Microsoft Project}. De esta
forma la información contenida por el archivo ``.mpp'' puede ser compartida en
la red por varios usuarios.
\end{itemize}

\subsubsection{Motivos de la elección}
Se eligió esta herramienta por diferentes motivos:
\begin{itemize}
\item Permite generar \emph{diagramas de Gantt}. Esto ayuda a determinar
cuáles son las tareas a desarrollar, quién se encargará de ello y cuándo ha de
hacerlo.
\item Es una herramienta web. Por lo que no necesitamos instalar ninguna
aplicación.
\item Puede integrarse en \emph{Google Docs} para permitir una mejor
colaboración entre los miembros del grupo.
\item Es una herramienta muy fácil de usar.
\end{itemize}

\section{Requisitos encontrados}


\section{Diagrama de \emph{Gantt} del proyecto}

%\section{Cambios realizados}

