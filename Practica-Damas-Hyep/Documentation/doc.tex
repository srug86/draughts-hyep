% Clase
\documentclass[11pt,a4paper,spanish,twoside]{report}

% Órdenes auxiliares
\input{inc/includes.tex}

%Para las secciones y las subseciones
\makeatletter
\renewcommand{\section}{
  \@startsection{section}{1}{0mm}{\baselineskip}
  {2.5mm}{\Large\bf}
}
\renewcommand{\subsection}{
  \@startsection{subsection}{2}{0mm}{2mm}
  {4.0mm}{\bf}
}
\renewcommand{\subsubsection}{
  \@startsection{subsubsection}{3}{0mm}{2mm}
  {0.1mm}{\normalsize\bf\emph}
}
\makeatother

% Encabezado y pie de página
\encabezado

\setcounter{secnumdepth}{3}
\setcounter{tocdepth}{3}

\begin{document}

% Silabación extra
\input{inc/hyphenations.tex}

% Portada
\portada{Herramientas y entornos de programación}
{Desarrollo de una aplicación utilizando \emph{\textbf{c\#}} y
 \emph{\textbf{.net}}}
{Juego de damas}
{Sergio de la Rubia García-Carpintero \\Juan Miguel Torres Triviño}
{19 de Enero de 2012}

% Licencia
\licencia{Sergio de la Rubia García-Carpintero, Juan Miguel Torres Triviño}

% Índices
\tableofcontents
% \listoffigures
% \listoftables

%% INICIO DEL DOCUMENTO %%%%%%%%%%%%%%%%%%%%%%%%%%%%%%%%%%%%%%%%%%%%%%%%%

\chapter{Introducción}

El objetivo de la práctica es conocer diferentes metodologías, tecnologías y
patrones para el diseño y construcción de softwares. Para ello se va
a realizar una aplicación que implementa el ya conocido juego de las damas. 

Se pretende que la aplicación permita a los usuarios jugar tanto en modo local 
como en modo red, siendo esta última una conexión uno a uno entre dos jugadores
o usuarios. Por lo tanto, el desarrollo de la aplicación se descompone en las 
siguientes etapas:

\begin{itemize}
\item \textbf{Definición de requisitos}: se realiza una análisis de las 
funcionalidades de la aplicación que tiene como resultado la lista de 
requisitos.
\item \textbf{Diseño estructural y funcional}: se realizan diversos diagramas a
partir de los requisitos finales para así entender mejor el funcionamiento de la
aplicación.
\item \textbf{Implementación}: se desarrolla un primer prototipo y a partir de 
éste se procede a implementar la aplicación completa.
\end{itemize}

Con la realización de esta práctica se pretende conocer y asimilar los 
objetivos básicos de \emph{Herramientas y Entornos de Programación}:

\begin{itemize}
\item Utilizar adecuadamente herramientas, lenguajes y entornos de programación
como \emph{Visual Studio}, \emph{Expression Blend}, \emph{.NET} y \emph{C\#}.
\item Adquirir una cierta cultura general respecto de métodos y tecnologías
para el desarrollo del software para las herramientas anteriormente 
mencionadas.
\end{itemize}


\chapter{FASE 1: Definición de requisitos}
\section{Herramientas utilizadas}
Para realizar la definición de requisitos se barajaron herramientas como:
\emph{Rational RequisitePro}, \emph{Visual Paradigm for ULM} y \emph{Gantter}.
Finalmente, se decidió que esta última sería la más adecuada para realizar
nuestro cometido.

\subsection{\emph{Gantter}}
Es una herramienta libre de gestión de proyectos basada en web que está
disponible en la página \url{http://gantter.com}.

\subsubsection{Características}
\emph{Gantter} posee las siguientes características:
\begin{itemize}
\item Herramienta de planificación de proyectos.
\item Permite establecer los plazos, los recursos, los costes y las tareas de
nuestro proyecto, así como generar con éstos \emph{diagramas de gantt}, 
calendarios, establecer la ruta crítica, etc.
\item Basada en web. Por lo que no hace falta instalar ningún software en
nuestra máquina. Para poder ejecutar la aplicación, la propia página web
ofrece los plugins necesarios para los principales navegadores: \emph{Chrome},
\emph{Firefox}, \emph{Explorer}, \emph{Opera} y \emph{Safari}.
\item Puede integrarse a \emph{Google Docs}. Esto permite que la información
pueda compartirse y actualizarse de forma colaborativa y distribuida.
\item Permite importar proyectos creados en \emph{Microsoft Project}. De esta
forma la información contenida por el archivo ``.mpp'' puede ser compartida en
la red por varios usuarios.
\end{itemize}

\subsubsection{Motivos de la elección}
Se eligió esta herramienta por diferentes motivos:
\begin{itemize}
\item Permite generar \emph{diagramas de Gantt}. Esto ayuda a determinar
cuáles son las tareas a desarrollar, quién se encargará de ello y cuándo ha de
hacerlo.
\item Es una herramienta web. Por lo que no necesitamos instalar ninguna
aplicación.
\item Puede integrarse en \emph{Google Docs} para permitir una mejor
colaboración entre los miembros del grupo.
\item Es una herramienta muy fácil de usar.
\end{itemize}

\section{Requisitos encontrados}


\section{Diagrama de \emph{Gantt} del proyecto}

%\section{Cambios realizados}



\chapter{FASE 2: Diseño estructural y funcional}
\section{Herramientas utilizadas}
Para realizar el diseño estructural y funcional de la aplicación se analizaron
herramientas como: \emph{Kivio}, \emph{UML Studio} o \emph{Visual Paradigm for
UML}. Al final, se decidió que \emph{Visual Paradigm for UML} sería la
herramienta más idónea para realizar esta fase.

\subsection{\emph{Visual Paradigm for UML}}
Es una herramienta \emph{CASE} multiplataforma que soporta \emph{UML 2},
\emph{SysML} y la notación y modelado de procesos de negocio (\emph{BPMN}).

\subsubsection{Características}
\emph{Visual Paradigm} posee las siguientes características:
\begin{itemize}
\item Diagramas de diseño automático sofisticado. 
\item Entorno de modelado visual. 
\item 13 tipos de diagramas UML (incluidos de clases y secuencia). 
\item Posibilidad de generar documentación y reportes \emph{UML}.
\item Generación automática de código en multitud de lenguajes de programación.
\item Ingeniería inversa.
\item Licencia propietaria (con una edición libre: la \emph{Free Community
Edition}).
\end{itemize}

\subsubsection{Motivos de la elección}
Se eligió esta herramienta por los siguientes motivos:
\begin{itemize}
\item Años de experiencia, utilizada en otras asignaturas. 
\item Gran poder descriptivo, mejor diseño y mejor implementación que otras 
herramientas analizadas. 
\item Permite la generación automática de código en \emph{C\#} a partir de los
diagramas diseñados.
\item Permite realizar ingeniería inversa. 
\item La ESI nos facilita una licencia para poder utilizar la edición
\emph{VP-UML Standard Edition}.
\end{itemize}

\section{Diagramas elaborados}
Con esta herramienta se elaboró el diagrama de clases de la aplicación y los
principales diagramas de secuencia.

\subsection{Diagramas de clases}
La figura \ref{img:Classes}, muestra un esquema general de los paquetes y
las clases que componen la implementación de la aplicación.

\begin{sidewaysfigure}
\imagen{Classes.pdf}{20}{Diagrama de clases}{img:Classes}
\end{sidewaysfigure}

La figura \ref{img:Presentation}, muestra en detalle las clases que componen el
paquete \emph{presentation}. Estas clases implementan las interfaces gráficas
de la aplicación.

\imagen{Presentation.pdf}{13}{Clases del paquete \emph{Presentation}}
{img:Presentation}

La figura \ref{img:Domain}, muestra en detalle las clases que componen el
paquete \emph{presentation}. Estas clases implementan la parte lógica de la
aplicación.

\imagen{Domain.pdf}{12}{Clases del paquete \emph{Domain}}{img:Domain}

La figura \ref{img:Communications}, muestra en detalle las clases que componen
el paquete \emph{presentation}. Estas clases implementan la parte de
comunicaciones entre la aplicación, la base de datos distribuida y otros
jugadores de la red.

\imagen{Communications.pdf}{10}{Clases del paquete \emph{Communications}}
{img:Communications}

\subsection{Diagramas de secuencia}
La figura \ref{img:AddUser} muestra el diagrama de secuencia del caso de uso
\emph{añadir un usuario a la base de datos}. El usuario especifica su nombre,
contraseña y avatar y se la envía a la base de datos distribuida.

\begin{sidewaysfigure}
\imagen{AddUser.pdf}{20}{Diagrama de secuencia de \emph{añadir un usuario a
la base de datos}}{img:AddUser}
\end{sidewaysfigure}

La figura \ref{img:LocalMode} muestra el diagrama de secuencia del caso de uso
\emph{iniciar una partida local contra la CPU}. El usuario selecciona
un nombre y un avatar para el jugador humano y un nombre y el avatar de la CPU
para el jugador virtual.

\imagen{LocalMode.pdf}{13}{Diagrama de secuencia de \emph{iniciar una partida
local contra la CPU}}{img:LocalMode}

La figura \ref{img:LocalPlay} muestra el diagrama de secuencia del caso de uso
\emph{realizar jugada en una partida local contra la CPU}. El jugador humano 
realiza el movimiento de una ficha y la CPU le responde con otra jugada.

\begin{sidewaysfigure}
\imagen{LocalPlay.pdf}{20}{Diagrama de secuencia de \emph{realizar jugada en
una partida local contra la CPU}}{img:LocalPlay}
\end{sidewaysfigure}

La figura \ref{img:Login} muestra el diagrama de secuencia del caso de uso
\emph{autentificarse para jugar en red}. El usuario introduce su nombre y
contraseña, se accede a la base de datos para comprobar que existe dicho
usuario y se comprueba si las contraseñas coinciden.

\begin{sidewaysfigure}
\imagen{Login.pdf}{20}{Diagrama de secuencia de \emph{autentificarse para
jugar en red}}{img:Login}
\end{sidewaysfigure}

La figura \ref{img:NetMode} muestra el diagrama de secuencia del caso de uso
\emph{iniciar una partida en red}. El usuario crea una partida y espera a que
otro jugador se una a ella. Cuando esto sucede, el usuario puede pulsar el
botón ``Empezar'' para comenzar la partida en red.

\begin{sidewaysfigure}
\imagen{NetMode.pdf}{20}{Diagrama de secuencia de \emph{iniciar una partida
en red}}{img:NetMode}
\end{sidewaysfigure}

La figura \ref{img:NetPlay} muestra el diagrama de secuencia del caso de uso
\emph{realizar jugada en una partida en red}. El usuario realiza el movimiento
de una ficha y se lo envía mediante la conexión TCP al oponente. Ambos
actualizan sus tableros.

\begin{sidewaysfigure}
\imagen{NetPlay.pdf}{20}{Diagrama de secuencia de \emph{realizar jugada en
una partida en red}}{img:NetPlay}
\end{sidewaysfigure}

%\section{Cambios realizados}



\chapter{FASE 3: Prototipos}
\section{Herramientas utilizadas}
Para desarrollar los prototipos de la aplicación y la aplicación en sí, se han
utilizado las dos herramientas que se han venido utilizando en las sesiones
prácticas de la asignatura: \emph{MS Visual Studio} y
\emph{MS Expression Blend}.

\subsection{\emph{Microsoft Visual Studio 2008}}
Es el \emph{IDE} utilizado en esta asignatura para desarrollar aplicaciones
\emph{WPF} (\emph{Windows Presentation Foundation}) en \emph{Visual C\#}.
Tiene las siguientes características:
\begin{itemize}
\item Tiene licencia propietaria y es soportado por las plataformas
\emph{Windows}.
\item Además de \emph{Visual C\#}, soporta otros lenguajes de programación
tales como \emph{Visual C++}, \emph{Visual J\#}, \emph{ASP.NET} y
\emph{Visual Basic .NET}, entre otros.
\item Permite desarrollar aplicaciones, sitios y aplicaciones web, así como
servicios web en cualquier entorno que soporte la plataforma \emph{.NET}. Así
se pueden crear aplicaciones que se intercomuniquen entre estaciones de
trabajo, páginas web y dispositivos móviles.
\item Permite la creación de soluciones multiplataforma adaptadas para
funcionar con las diferentes versiones de \emph{.NET Framework}: 2.0
(incluido con \emph{Visual Studio 2005}), 3.0 (incluido en
\emph{Windows Vista}) y 3.5 (incluido con \emph{Visual Studio 2008}).
\end{itemize}

Con esta aplicación se ha generado la solución \emph{Draughts.sln} y a partir
de ahí, se han ido incorporando los paquetes y las clases que forman parte del
proyecto.

Además, para facilitar el desarrollo colaborativo de la aplicación, se ha
utilizado un complemento llamado \emph{\textbf{AnkhSVN}} (\emph{Subversion
Support for Visual Studio}), que permite mantener un control de versiones de
los códigos fuente del proyecto. Así, todos los miembros del grupo tendrán a su 
disposición siempre la última versión del código fuente del proyecto y podrán 
compartir con los demás los cambios que van realizando.

\subsection{\emph{Microsoft Expression Blend 2}}
Es una herramienta de diseño que facilita la implementación de interfaces
gráficas de usuario definidas en \emph{XAML} y \emph{Silverlight}.

Tiene las siguientes características:
\begin{itemize}
\item Permite trabajar de forma sincronizada con proyectos de \emph{Visual
Studio 2008}.
\item Ofrece un amplio abanico de posibilidades para el fácil desarrollo de 
interfaces gráficas originales e innovadoras.
\item Permite generar animaciones con los elementos de la interfaz.
\end{itemize}

Con esta aplicación se han generado las interfaces gráficas de usuario de la 
aplicación. Su potencia descriptiva ha permitido desarrollar unas interfaces
con un diseño innovador y de una forma más fácil y más rápida que si hubiera
tenido que hacerse de forma manual o mediante las herramientas que proporciona
el propio \emph{Visual Studio} para tal efecto.

\section{Prototipo}
Haciendo uso de las herramientas nombradas anteriormente se ha procedido al
desarrollo de la aplicación. El 12 de diciembre de 2011 el estado del prototipo
de la aplicación desarrollado hasta entonces era el siguiente:

\subsection{Interfaz gráfica de usuario}
La interfaz gráfica estaba prácticamente terminada. Estas eran las ventanas
que se habían desarrollado:

\subsubsection{Ventana de inicio}
La ventana de inicio muestra cuatro opciones (ver figura \ref{img:proInit}):
\begin{itemize}
\item \textbf{Añadir usuario}: Sirve para abrir la ventana de creación de
jugadores.
\item \textbf{Iniciar partida}: Que nos lleva a la ventana de selección de
jugadores para disputar una partida local.
\item \textbf{Jugar en red}: Que abre la ventana de conexión, para concretar
una partida en red con otro jugador.
\item \textbf{Ranking}: Muestra el ranking con los mejores jugadores
registrados en el sistema.
\end{itemize}
Además, pulsando ``?'', en esta o en cualquiera de las otras ventana, se puede 
acceder a la ayuda de la aplicación.

\imagen{proInitWin.png}{10}{Ventana de inicio del prototipo de la aplicación.}
{img:proInit}

\subsubsection{Ventana de creación de jugadores}
La ventana de creación de jugadores (figura \ref{img:proProfile}) permite al
usuario crear nuevos perfiles de jugadores. Para ello, debe especificar el
nombre del jugador y seleccionar un avatar de entre los nueve avatares
disponibles. El jugador se creará pulsando el botón ``Aceptar''.

\imagen{proProfileWin.png}{10}{Ventana de creación de jugadores del prototipo
de la aplicación.}{img:proProfile}

\subsubsection{Ventana que muestra el ranking}
La ventana de ranking (figura \ref{img:proRanking}) muestra una lista con los
10 mejores jugadores registrados en el sistema. Para elaborar este ranking se
tienen en cuenta las partidas ganadas, empatadas y perdidas de cada jugador. 
Además, en la parte derecha aparecen el nombre y el avatar del mejor jugador 
actual.

\imagen{proRankingWin.png}{10}{Ventana de ranking de jugadores del prototipo
de la aplicación.}{img:proRanking}

\subsubsection{Ventana de selección de jugadores para partida local}
Antes de iniciar una partida local, el usuario determinará cuáles son los 
jugadores que tomarán parte en ella, pudiendo elegir entre los jugadores 
registrados y la máquina. Para hacerlo, hará uso de las flechas de izquierda o 
derecha. El jugador de la izquierda jugará con las fichas rojas (inicia la 
partida) y el de la izquierda con las fichas blancas. Para comenzar la
partida, el usuario deberá seleccionar ``Empezar''.

La figura \ref{img:proSelect} muestra el aspecto de esta ventana.

\imagen{proSelectWin.png}{10}{Ventana de selección de jugadores para partida
local.}{img:proSelect}

\subsubsection{Ventana de selección de jugadores para partida en red}
A través de esta ventana \ref{img:proConnect} uno de los usuarios se encargará
de crear una partida (especificando un puerto, su dirección IP, su jugador y
dándole al botón ``Crear partida'') y otro (desde su máquina) se encargará de 
unirse a esta (especificando el mismo puerto e IP que el primero, un jugador y 
dándole al botón ``Unirme a partida''). Cuando la conexión entre ambos sea 
efectiva, el avatar y el nombre del otro se harán visibles y se permitirá
chatear. Para comenzar la partida, ambos deberán pulsar el botón de
``Empezar''.

\imagen{proConnectWin.png}{10}{Ventana de selección de jugadores para partida
en red.}{img:proConnect}

\subsubsection{Ventana de juego}
La ventana de juego, tanto para juegos en red como para juegos locales, tiene
un aspecto similar al de la figura \ref{img:proGame}.

\imagen{proGameWin.png}{10}{Ventana de juego del prototipo de la aplicación.}
{img:proGame}

La parte izquierda muestra el tablero de juego. Al pulsar sobre una de las piezas 
(cuando tenemos el turno) se iluminan las casillas a las que podemos saltar.

En la parte derecha, se muestran los nombres y avatares de los jugadores que 
participan; y una pequeña frase que indica a quién pertenece el turno de juego o 
quién es el ganador.

\subsection{Lógica de la aplicación}
La lógica de la aplicación implementada hasta la fecha estaba en su fase
inicial. Sólamente estaba implementada la funcionalidad que permitía
navegar entre unas ventanas y otras.

Por otro lado, el resto de clases del proyecto están añadidas a la solución
del proyecto, pero faltaba por implementar la mayor parte de la funcionalidad 
necesaria para poder llevar a cabo una partida, tanto local como en red.

\section{Cambios realizados}



\chapter{Manual de usuario}
\section{Ventana inicial}
Cuando un usuario inicia la aplicación \emph{Draughts} la primera ventana de la
 interfaz que le aparece es la mostrada en la figura \ref{img:ini}. En esta 
ventana aparecen las siguientes opciones:
\begin{itemize}
\item \textbf{Añadir Usuario}: esta opción permite crear un usuario para poder 
jugar en red.
\item \textbf{Iniciar Partida}: permite jugar una partida entre jugadores 
humanos o contra la CPU en modo local, es decir, no hace falta registrarse en 
la aplicación.
\item \textbf{Jugar en Red}: esta opción permite que un jugador o usuario de la 
aplicación juegue en red con otro usuario.
\item \textbf{Ranking}: muestra una ventana con el ranking actual de la base de
 datos.
\item \textbf{X}: cierra la aplicación.
\item \textbf{?}: esta opción abre un archivo generado por \emph{SandCastle} en
 formato de ayuda donde se pueden obsevar una explicación de los métodos e 
información importante, como se puede ver en la figura \ref{img:ayu}.
\end{itemize}

\imagen{Inicial.png}{8}{Ventana inicial de la aplicación.}{img:ini}
\imagen{Ayuda.png}{8}{Ventana para mostar la ayuda.}{img:ayu}

\section{Partida local}
Después de elegir la opción \emph{Iniciar Partida}, la cual va a permitir jugar
 una partida de damas sin la necesidad de registrarse.
\subsection{Seleccionar jugadores}
La primera ventana que aparece es la mostrada en la figura \ref{img:sel} que da
la posibilidad de elegir cualquier avatar y nombre, tanto para un jugador1 
humano como un jugador2, este último puede ser cualquier jugador humano o puede
ser la CPU. En esta ventana tiene las siguientes opciones:
\begin{itemize}
\item \textbf{Empezar}: esta opción permite abrir una partida de damas con los 
jugadores seleccionados.
\item \textbf{Atrás}: vuelve a la ventana anterior.
\item \textbf{<}: permite seleccionar el avatar anterior tanto para el jugador1
 como para el jugador2.
\item \textbf{>}: permite seleccionar el avatar siguiente tanto para el 
jugador1 como para el jugador2.
\end{itemize}
\imagen{SeleccionarJugadores.png}{8}{Ventana de selección de jugadores.}{img:sel}

\subsection{Inicio de la partida}
Una vez pulsada la opción \emph{Empezar} se muestra la ventana que aparece en 
la figura \ref{img:jug}, en ésta se muestran la información de los jugadores 
seleccionados anteriormente y el tablero de juego, donde transcurre la partida
de damas.
\imagen{Juego.png}{8}{Ventana de juego inicialmente.}{img:jug}

\subsection{Movimientos durante la partida}
Durante el transcurso de la partida un jugador puede seleccionar cualquier 
casilla del tablero que contenga una ficha y la ventana muestra los posibles
movimientos a realizar. A continuación el jugador pulsa sobre una casilla 
iluminada con lo que la ficha procede a moverse hacia esa posición elegida. Un 
ejemplo de posibles movientos es el que se puede ver en la figura \ref{img:mov}.
\imagen{MovJuego.png}{8}{Ventana de juego durante un movimiento.}{img:mov}

\subsection{Final de la partida}
Cuando la partida finalice porque el jugador1 ha ganado, el jugador2 ha ganado o
se ha producido un empate, la ventana muestra dicho resultado como se puede ver
en la figura \ref{img:fin}.
\imagen{FinJuego.png}{8}{Ventana de juego cuando acabo la partida.}{img:fin}

\section{Añadir usuario}
Si el jugador selecciona la opción \emph{Añadir Usuario} en la ventana inicial
se muestra la ventana que se puede ver en la figura \ref{img:usu1}. En dicha
ventana el usuario debe indicar un nombre, una contraseña y un avatar, 
seleccionando dentro los mostrados en la ventana. Además esta ventana tiene
las siguientes opciones:
\imagen{CrearJugador1.png}{8}{Ventana de añadir usuario sin datos.}{img:usu1}
\begin{itemize}
\item \textbf{Aceptar}: esta opción crea el usuario en la aplicación para poder 
loguearse más tarde.
\item \textbf{Cancelar}: vuelve a la ventana inicial.
\end{itemize}
Cuando se pulsa en una avatar de los posibles, se muestra dicho avatar 
seleccionado en la recuadro de mayor tamaño, esta característica de la ventana
se puede ver en la figura \ref{img:usu2}.
\imagen{CrearJugador2.png}{8}{Ventana de añadir usuario con datos.}{img:usu2}
En el caso de que no se rellene un campo, es decir, esté vacío o que el nombre
introducido ya exista en la base de datos, la ventana muestra un mensaje de 
información diciendo el problema e impidiendo crear el usuario.

\section{Partida en red}
Cuando un usuario elige la opción \emph{Jugar en Red} puede empezar a loguearse
 y posteriormente a realizar partidas de damas en red y comunicación por chat.
\subsection{\emph{Login}}
La primera ventana que aparece es la que se ve en la figura \ref{img:log}, donde
el usuario puede loguearse si previamente ha creado un usuario. En el caso de 
que el nombre introducido y/o la contraseña sean vacíos o erróneos, la ventana
muestra un mensaje de información detallando el problema.Esta ventana tiene las
siguientes opciones:
\begin{itemize}
\item \textbf{Aceptar}: comprueba si el login realizado es correcto.
\item \textbf{Cancelar}: vuelve a la ventana anterior.
\end{itemize}
\imagen{Login.png}{8}{Ventana del login para la aplicación.}{img:log}

\subsection{Establecer una conexión}
Cuando el jugador se ha logueado puede establecer conexión con otro jugador
logueado obviamente, para ello se muestra en la ventana de la figura 
\ref{img:con} las siguientes opciones:
\imagen{Conectarse.png}{8}{Ventana de conexión de jugadores.}{img:con}
\begin{itemize}
\item \textbf{Crear partida}: esta opción hace que el jugador que lo pulsa 
actúe como servidor de la partida.
\item \textbf{Unirse a la partida}: esta opción hace que el jugador que lo 
pulsa actúe como cliente de la partida.
\item \textbf{Empezar}: sólo aparece cuando dos jugadores han establecido 
conexión.
\item \textbf{Enviar}: permite enviar un mensaje al otro jugador (chat).
\end{itemize}
Para poder realizar las opciones de \emph{Empezar} y \emph{Enviar}, los 
jugadores han debido establerce una conexión, para ello uno de los jugadores
ha introducido la dirección y puerto del jugador servidor. Una vez establecida
la conexión se muestra la información de los jugadores y se puede jugar una 
partida como hablar en un chat, que se puede ver en la figura \ref{img:chat}.
\imagen{Chat.png}{8}{Ventana de muestra el chat.}{img:chat}

\section{Ranking}
Si el usuario elige la opción \emph{Ranking}, le aparece la siguiente ventana de
la figura \ref{img:rank} en la que se muestra el ranking de los diez mejores 
jugadores de la aplicación como la información del mejor jugador.\\

\imagen{Ranking.png}{8}{Ventana que muestra el ranking.}{img:rank}




\end{document}
