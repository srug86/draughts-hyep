\section{Ventana inicial}
Cuando un usuario inicia la aplicación \emph{Draughts} la primera ventana de la
 interfaz que le aparece es la mostrada en la figura \ref{img:ini}. En esta 
ventana aparecen las siguientes opciones:
\begin{itemize}
\item \textbf{Añadir Usuario}: esta opción permite crear un usuario para poder 
jugar en red.
\item \textbf{Iniciar Partida}: permite jugar una partida entre jugadores 
humanos o contra la CPU en modo local, es decir, no hace falta registrarse en 
la aplicación.
\item \textbf{Jugar en Red}: esta opción permite que un jugador o usuario de la 
aplicación juegue en red con otro usuario.
\item \textbf{Ranking}: muestra una ventana con el ranking actual de la base de
 datos.
\item \textbf{X}: cierra la aplicación.
\item \textbf{?}: esta opción abre un archivo generado por \emph{SandCastle} en
 formato de ayuda donde se pueden obsevar una explicación de los métodos e 
información importante, como se puede ver en la figura \ref{img:ayu}.
\end{itemize}

\imagen{Inicial.png}{8}{Ventana inicial de la aplicación.}{img:ini}
\imagen{Ayuda.png}{8}{Ventana para mostar la ayuda.}{img:ayu}

\section{Partida local}
Después de elegir la opción \emph{Iniciar Partida}, la cual va a permitir jugar
 una partida de damas sin la necesidad de registrarse.
\subsection{Seleccionar jugadores}
La primera ventana que aparece es la mostrada en la figura \ref{img:sel} que da
la posibilidad de elegir cualquier avatar y nombre, tanto para un jugador1 
humano como un jugador2, este último puede ser cualquier jugador humano o puede
ser la CPU. En esta ventana tiene las siguientes opciones:
\begin{itemize}
\item \textbf{Empezar}: esta opción permite abrir una partida de damas con los 
jugadores seleccionados.
\item \textbf{Atrás}: vuelve a la ventana anterior.
\item \textbf{<}: permite seleccionar el avatar anterior tanto para el jugador1
 como para el jugador2.
\item \textbf{>}: permite seleccionar el avatar siguiente tanto para el 
jugador1 como para el jugador2.
\end{itemize}
\imagen{SeleccionarJugadores.png}{8}{Ventana de selección de jugadores.}{img:sel}

\subsection{Inicio de la partida}
Una vez pulsada la opción \emph{Empezar} se muestra la ventana que aparece en 
la figura \ref{img:jug}, en ésta se muestran la información de los jugadores 
seleccionados anteriormente y el tablero de juego, donde transcurre la partida
de damas.
\imagen{Juego.png}{8}{Ventana de juego inicialmente.}{img:jug}

\subsection{Movimientos durante la partida}
Durante el transcurso de la partida un jugador puede seleccionar cualquier 
casilla del tablero que contenga una ficha y la ventana muestra los posibles
movimientos a realizar. A continuación el jugador pulsa sobre una casilla 
iluminada con lo que la ficha procede a moverse hacia esa posición elegida. Un 
ejemplo de posibles movientos es el que se puede ver en la figura \ref{img:mov}.
\imagen{MovJuego.png}{8}{Ventana de juego durante un movimiento.}{img:mov}

\subsection{Final de la partida}
Cuando la partida finalice porque el jugador1 ha ganado, el jugador2 ha ganado o
se ha producido un empate, la ventana muestra dicho resultado como se puede ver
en la figura \ref{img:fin}.
\imagen{FinJuego.png}{8}{Ventana de juego cuando acabo la partida.}{img:fin}

\section{Añadir usuario}
Si el jugador selecciona la opción \emph{Añadir Usuario} en la ventana inicial
se muestra la ventana que se puede ver en la figura \ref{img:usu1}. En dicha
ventana el usuario debe indicar un nombre, una contraseña y un avatar, 
seleccionando dentro los mostrados en la ventana. Además esta ventana tiene
las siguientes opciones:
\imagen{CrearJugador1.png}{8}{Ventana de añadir usuario sin datos.}{img:usu1}
\begin{itemize}
\item \textbf{Aceptar}: esta opción crea el usuario en la aplicación para poder 
loguearse más tarde.
\item \textbf{Cancelar}: vuelve a la ventana inicial.
\end{itemize}
Cuando se pulsa en una avatar de los posibles, se muestra dicho avatar 
seleccionado en la recuadro de mayor tamaño, esta característica de la ventana
se puede ver en la figura \ref{img:usu2}.
\imagen{CrearJugador2.png}{8}{Ventana de añadir usuario con datos.}{img:usu2}
En el caso de que no se rellene un campo, es decir, esté vacío o que el nombre
introducido ya exista en la base de datos, la ventana muestra un mensaje de 
información diciendo el problema e impidiendo crear el usuario.

\section{Partida en red}
Cuando un usuario elige la opción \emph{Jugar en Red} puede empezar a loguearse
 y posteriormente a realizar partidas de damas en red y comunicación por chat.
\subsection{\emph{Login}}
La primera ventana que aparece es la que se ve en la figura \ref{img:log}, donde
el usuario puede loguearse si previamente ha creado un usuario. En el caso de 
que el nombre introducido y/o la contraseña sean vacíos o erróneos, la ventana
muestra un mensaje de información detallando el problema.Esta ventana tiene las
siguientes opciones:
\begin{itemize}
\item \textbf{Aceptar}: comprueba si el login realizado es correcto.
\item \textbf{Cancelar}: vuelve a la ventana anterior.
\end{itemize}
\imagen{Login.png}{8}{Ventana del login para la aplicación.}{img:log}

\subsection{Establecer una conexión}
Cuando el jugador se ha logueado puede establecer conexión con otro jugador
logueado obviamente, para ello se muestra en la ventana de la figura 
\ref{img:con} las siguientes opciones:
\imagen{Conectarse.png}{8}{Ventana de conexión de jugadores.}{img:con}
\begin{itemize}
\item \textbf{Crear partida}: esta opción hace que el jugador que lo pulsa 
actúe como servidor de la partida.
\item \textbf{Unirse a la partida}: esta opción hace que el jugador que lo 
pulsa actúe como cliente de la partida.
\item \textbf{Empezar}: sólo aparece cuando dos jugadores han establecido 
conexión.
\item \textbf{Enviar}: permite enviar un mensaje al otro jugador (chat).
\end{itemize}
Para poder realizar las opciones de \emph{Empezar} y \emph{Enviar}, los 
jugadores han debido establerce una conexión, para ello uno de los jugadores
ha introducido la dirección y puerto del jugador servidor. Una vez establecida
la conexión se muestra la información de los jugadores y se puede jugar una 
partida como hablar en un chat, que se puede ver en la figura \ref{img:chat}.
\imagen{Chat.png}{8}{Ventana de muestra el chat.}{img:chat}

\section{Ranking}
Si el usuario elige la opción \emph{Ranking}, le aparece la siguiente ventana de
la figura \ref{img:rank} en la que se muestra el ranking de los diez mejores 
jugadores de la aplicación como la información del mejor jugador.\\

\imagen{Ranking.png}{8}{Ventana que muestra el ranking.}{img:rank}


